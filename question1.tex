%% TeXEdit: SRC=LATEX;DEST=DVI
\documentclass[12pt,a4paper]{article}
\usepackage[utf8]{inputenc}
\usepackage[english,russian]{babel}
%\usepackage{amsmath}
\usepackage{amsthm}
%\usepackage{amsfonts}
 \usepackage{amssymb}
  \usepackage{latexsym}
\usepackage[utf8]{inputenc}
\usepackage[russian]{babel}
\usepackage{enumitem}
\usepackage{xcolor}
\usepackage[unicode,pdftex,colorlinks,linkcolor=blue,draft=false,breaklinks,bookmarksopen]{hyperref}
\let\alph\asbuk

\voffset=-25.4mm
\topmargin=20mm
\headheight=0mm
\headsep=0mm
\textheight=247mm

\hoffset=-25.4mm
\oddsidemargin=20mm
\textwidth=170mm

\hypersetup{pdftitle={Дискретная математика, часть 1}}
\hypersetup{pdfauthor={Дудаков Сергей Михайлович}}
\hypersetup{pdfsubject={Вопросы к экзамену}}
\hypersetup{pdfkeywords={дискретная математика, множества, комбинаторика, математическая индукция, булевы функции, логика предикатов}}
\hypersetup{pdfdisplaydoctitle,pdflang=ru,citecolor=red}


\begin{document}
\title{Дискретная математика}
\author{Вопросы к экзамену}
\date{1 курс, 1 семестр}
\maketitle
\thispagestyle{empty}

\begin{enumerate}

\item
Множества, операции над ними. Объединение, пересечение, разность, декартово произведение, декартова степень.
Подмножества.
Основные свойства теоретико"=множественных операций.
Парадокс Рассела.
\item
Функции и отношения. Мощность множества. Равномощные множества.
Теорема Кантора о мощности множества всех подмножеств.
\item
Принцип математической индукции. Прямая и обратная индукция.
\item
Основные элементы комбинаторики: перестановки,
размещения с повторениями и без повторений, сочетания.
Формулы для количества объектов каждого вида.
\item
Булевы функции. Задание булевых функций таблицами. Лексикографический порядок.
\item
Формулы логики высказываний. Синтаксис и семантика.
Основные булевы связки: конъюнкция, дизъюнкция, импликация, отрицание, сложение по модулю 2.
\item
Основные эквивалентности логики высказываний.
\item
Элементарные конъюнкции и дизъюнкции.
Дизъюнктивные и конъюнктивные нормальные формы. Совершенные ДНФ.
Построение КНФ и ДНФ по булевой функции.
\item
Сокращённая ДНФ. Построение сокращённой ДНФ методом Блейка.
\item
Многочлены Жегалкина.
Построение многочлена Жегалкина по булевой функции методом неопределённых коэффициентов.
\item
Суперпозиция булевых функций. Замкнутые относительно
суперпозиции классы булевых функций. Полная системы булевых функций.
\item
Классы булевых функций, сохраняющих 0, сохраняющих 1, линейных, самодвойственных, монотонных.
\item
Теорема Поста о полноте множества булевых функций.
\item
Формулы логики предикатов. Синтаксис.
Атомные формулы, кванторы, свободные и связанные переменные, область действия квантора.
\item
Семантика логики предикатов. Интерпретации.
Значение формул в интерпретации. Следование и эквивалентность.
Тождественная истинность и тождественная ложность.
\item
Основные эквивалентности логики предикатов. Предварённая форма.
\end{enumerate}
\end{document}
